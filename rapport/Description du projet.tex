\chapter*{Description du projet}
\addcontentsline{toc}{chapter}{Description du projet}
\section{Objectif du projet}
	 Ce projet a pour objectif de réaliser un distributeur automatique de nourriture pour chat contrôlé à distance via une application mobile (Android) développée en Kotlin, utilisant un microcontrôleur (ESP32), plusieurs capteurs, un servo-moteur pour la partie matérielle et Firebase comme base de données en temps réel pour la synchronisation des données.

\section{Besoins des utilisateurs}

Voici la liste des besoins des utilisateurs :

\begin{itemize}
\item     Programmation des repas : Définir des horaires et des quantités de nourriture.

\item     Contrôle manuel : Distribuer de la nourriture à la demande via l'application.

\item     Notifications : Alertes lorsque la nourriture est distribuée ou lorsque le niveau est faible.

\item     Historique : Suivi des distributions passées.

\item     Sécurité : Authentification des utilisateurs pour éviter les accès non autorisés.

\item  Réglage des quantités de quantité de nourriture pour chaque distribution

\item 
\end{itemize}
    
\section{Ressource nécessaire}

\pagenumbering{arabic}
\setcounter{page}{1}       





