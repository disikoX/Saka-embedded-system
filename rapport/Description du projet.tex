\chapter{Description du projet}
\addcontentsline{toc}{chapter}{Description du projet}
\section{Formulation}
Ce projet consiste à concevoir un distributeur automatique de nourriture pour chat, qui peut être contrôlé à distance via une application mobile. 
\section{Objectif du projet}
	L'objectif est de permettre aux propriétaires de chats de nourrir leurs animaux même lorsqu'ils sont absents, en programmant des horaires et des quantités de nourriture à distribuer. Le système doit également offrir des fonctionnalités de contrôle manuel, de notifications et d'historique des distributions.
\section{Besoins des utilisateurs}

Voici la liste des besoins des utilisateurs :

\begin{itemize}
	\item Planification des repas : Les utilisateurs devraient pouvoir définir à l'avance les horaires et les quantités de nourriture à distribuer.
	\item Distribution manuelle : Il est important de permettre aux utilisateurs de distribuer de la nourriture à la demande via l'application, en dehors des horaires programmés.
	\item Notifications : des alertes devraient être générées lorsque la nourriture est distribuée ou lorsque le niveau de nourriture est bas, afin de rester informés en temps réel.
	\item Historique des distributions : Un suivi des distributions pour que les utilisateurs puissent consulter les repas précédemment servis.
	\item Sécurité des accès : une authentification sécurisée pour éviter tout accès non autorisé à l'application.
	\item Réglage des quantités : permettre aux utilisateurs de définir précisément la quantité de nourriture à distribuer à chaque repas.
\end{itemize}
    
\section{Moyen nécessaire à la réalisation du projet}
	\subsection{Moyens Matériel}
	Pour réaliser ce projet, on aura besoin des éléments suivants :
		\begin{itemize}
			\item \textit{1 microcontrôleur ESP32} pour la connectivité Wi-Fi et le contrôle des composants.
			\item \textit{1 servo-moteur} pour contrôler la distribution de la nourriture.
			\item \textit{2 capteurs de poids} permettant de mesurer la quantité de nourriture distribuée ainsi que le reste de nourriture dans le reservoir.
			\item \textit{Des bois et des vis} pour assembler le distributeur.
			\item \textit{2 LED} pour indiquer l'état du distributeur (distribution en cours, niveau de nourriture bas).
		\end{itemize}
	\subsection{Moyens Logiciel}
	Pour le développement du logiciel, les outils suivants seront nécessaires :
		\begin{itemize}
			\item \textit{IDE Arduino} : pour la programmation de l'ESP32.
			\item \textit{Android Studio} : pour le développement de l'application mobile avec Kotlin et jetpack Compose.
			\item \textit{Firebase Realtime Database} : pour le stockage des données des utilisateurs et des configurations de distribution.
			\item \textit{Serveur websocket (Nodejs)} : pour gérer les requêtes entre l'ESP32 et la base de données.
			\item \textit{Git} : pour la gestion des versions du code source.
			\item \textit{WokWi} : pour la simulation des comportements des matériels éléctroniques.
			\item \textit{LatexMaker} : pour la rédaction du rapport de projet.
		\end{itemize}
	
		\subsection{Moyens Humains}
	Pour la réalisation de ce projet, une équipe de 4 personnes est nécessaire, avec les rôles suivants :
		\begin{itemize}
			\item \textbf{2 développeurs} : responsables de la programmation de l’ESP32, de l’intégration des capteurs et du développement des communications WebSocket.
			\item \textbf{2 développeurs} : chargés de la conception et du développement de l’application mobile.
		\end{itemize}

\pagenumbering{arabic}
\setcounter{page}{1}       





