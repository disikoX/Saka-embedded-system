
\pagestyle{fancy}
\fancyhead{} % clear all header fields
\fancyhead[LO,CE]{Tests et validation}
\fancyhead[RO,LE]{2024-2025}
\chapter{Tests et validation}
\section{Application mobile}

\subsection{La page d'accueil}
La figure~\ref{fig:interface-accueil} montre la page d'accueil de l'application. Cette interface permet à l’utilisateur de déclencher une distribution manuelle et de visualiser le distributeur sélectionné, la quantité de ration actuelle, ainsi que les alertes causées par le seuil critique configuré.

\begin{minipage}{\linewidth}
  \centering
  \includegraphics[scale=0.35]{page-acceuil.jpeg}
  \captionof{figure}{Interface d'accueil de l'application mobile}
  \label{fig:interface-accueil}
\end{minipage}

\subsection{Planning de Distribution}
La figure~\ref{fig:interface-planning} montre la page de planification. Elle permet à l'utilisateur de voir les prochaines distributions prévues. Il peut activer ou désactiver les heures de distribution, mettre temporairement en pause toutes les distributions, ou reprendre manuellement le fonctionnement. Un indicateur affiche aussi l’état de connexion du distributeur.

\begin{minipage}{\linewidth}
  \centering
  \includegraphics[scale=0.35]{page-planning.jpeg}
  \captionof{figure}{Page de planification des distributions}
  \label{fig:interface-planning}
\end{minipage}

\subsection{Liste des historiques}
La figure~\ref{fig:interface-historique} montre la page d’historique. Elle affiche toutes les distributions de nourriture effectuées, avec l’heure, la date, la quantité distribuée et le statut de réussite. L’utilisateur peut consulter donc l’historique d’aujourd’hui ou de toute la semaine.

\begin{minipage}{\linewidth}
  \centering
  \includegraphics[scale=0.35]{page-historique.jpeg}
  \captionof{figure}{Historique des distributions de nourriture}
  \label{fig:interface-historique}
\end{minipage}

\subsection{Interface de réglages}
Cette section de l’application, illustré dans la figure~\ref{fig:interface-reglage}, permet à l’utilisateur de configurer des paramètres comme la quantité de ration à distribuer ou le seuil critique. Les champs de saisie permettent d’entrer ces valeurs manuellement, puis de les valider. Ces options donnent plus de contrôle à l'utilisateur sur la gestion de la distribution de nourriture.

\begin{minipage}{\linewidth}
  \centering
  \includegraphics[scale=0.35]{page-reglage.jpeg}
  \captionof{figure}{Historique des distributions de nourriture}
  \label{fig:interface-reglage}
\end{minipage}

\section{Distributeur automatique}
En guise de test, puisqu'on ne peut pas montrer directement ici comment le matériel fonctionne, nous avons utiliser l' \verb|interface série (Serial)| de l'ESP32 pour montrer que le code de distribution s'exécute très bien.
\subsection{Distribution manuel}
\subsection{Distribution automatique des planning}