\documentclass[a4paper,12pt]{report}
\usepackage{titlesec}
\usepackage{tocloft} % pour la table des matières
\usepackage{pdfpages}
\usepackage[utf8]{inputenc}
\usepackage[backend=biber]{biblatex}
\addbibresource{biblio.bib}
\usepackage{hyperref}
\usepackage{abstract} % nous on a un abstract
\usepackage{tabularx}
\usepackage{graphicx}
\usepackage{xcolor}
\usepackage{array}
\usepackage{float}
\usepackage{amsmath}
\usepackage{multirow}
\usepackage{appendix}
\usepackage{adjustbox}
\usepackage{fancyhdr}
\usepackage{listings}
\usepackage{import}
\usepackage[nottoc,numbib]{tocbibind} %adding the bib to toc
\usepackage{rotating} %pour les tableaux
\usepackage{enumitem} % pour citer (mettre des tirets etc)
\usepackage[nounderscore]{syntax}
\usepackage[T1]{fontenc}
\usepackage[french]{babel} %comme ça tout tes list of figures deviendront direct en francais tables des
\usepackage[french]{nomencl}
\usepackage{xcolor,graphicx}
\newcounter{insertpages}
\usepackage[automake]{glossaries} %liste des abréviations
%\usepackage[acronym]{glossaries}
\graphicspath{{image/}} %path de tes images
\newcommand{\RomanNumeralCaps}[1]
{\MakeUppercase{\romannumeral #1}}
\usepackage[top=2cm, bottom=2cm, left=3cm, right=2cm]{geometry} % dimension de la page
\linespread{1.5}
\newenvironment{poliabstract}[1]
  {\renewcommand{\abstractname}{#1}\begin{abstract}}
  {\end{abstract}} %abstract sur 2 pages

\renewcommand{\acronymname}{Listes des abréviations}

\makeglossaries
%\newglossaryentry{maths}
%{
    %name=mathematics,
    %description={Mathematics is what mathematicians do}
%}


%%%%% style pour le code %%%%


% Definig a custom style:
\lstdefinestyle{mystyle}{
    backgroundcolor=\color{white},   
    commentstyle=\color{purple},
    keywordstyle=\color{blue},
    numberstyle=\tiny\color{gray},
    stringstyle=\color{purple},
    basicstyle=\ttfamily\footnotesize\bfseries,
    breakatwhitespace=false,         
    breaklines=true,                 
    captionpos=t,                    
    keepspaces=false,                 
    numbers=left,                    
    numbersep=5pt,                  
    showspaces=false,                
    showstringspaces=false,
    showtabs=false,                  
    tabsize=2
}

% -- Setting up the custom style:
\lstset{style=mystyle}

\lstset{
  style=mystyle,
%  frame= single,
%  linewidth=0.6\linewidth,
%  xleftmargin=12pt,
%  aboveskip=12pt,
%  belowskip=12pt
}

%%%%%%%%%%%%%%%%%%



%\loadglsentries{liste des abréviations}



\begin{document}
\renewcommand{\listfigurename}{Table des figures}
\newpage
\renewcommand{\listtablename}{Listes des tableaux}
\newpage
%\renewcommand\appendixpagename{Annexe}

\pagenumbering{gobble}


%\includepdf[pages={1}]{COUVERTURE.pdf} %car je ne peux pas inclure un fichier qui n'est pas déjà compilé

\null\newpage %saut de page

%\renewcommand*\contentsname{Sommaire}
%\includepdf[pages={1}]{couverture.pdf}
\tableofcontents{} %table des matières
%\include{Cahier des charges}

\pagenumbering{roman} %numérotation des pages


\listoffigures{}
\null\newpage
%\vspace{6cm}
\listoftables{}

%\printglossary[title={Nomenclature}]
%\addcontentsline{toc}{chapter}{Nomenclature}
%\vspace{7cm}
%\input{Liste des abréviations}








%% gesticulation c'est le header en haut 
\pagestyle{fancy}
\fancyhead{} % clear all header fields
\fancyhead[LO,CE]{Distributeur intelligent d'aliments pour chat}
\fancyhead[RO,LE]{2024-2025}
%\fancyfoot[C]{\thepage} % Exemple footer
%\renewcommand{\footrulewidth}{1pt} %  ligne pour le footer


 %prend le fichier .tex l'ajoute à la compilation
%\input{généralités}
%\input{ETUDE ET CONCEPTION}
%\input{REALISATION ET ESSAIS}
%\input{CONCLUSION GENERALE}

%pour afficher la bilbio
\nopagebreak
\printbibliography[heading=bibintoc,type=misc,title={Bibliographie}] 
 
\pagenumbering{Roman}
\setcounter{page}{8}
\nopagebreak
\printbibliography[heading=bibintoc,type=online,title={Webographie}]
\nopagebreak
\nopagebreak[4]
\addappheadtotoc
%\clearpage
%\include{ANNEXE}
%\printglossary[title=Glossaire, toctitle=Glossaire]
\null\newpage
%\renewcommand*\contentsname{Table des matières}
%\tableofcontents{} %table des matières
%\include{Cahier des charges}
\chapter*{Description du projet}
\addcontentsline{toc}{chapter}{Description du projet}
\section{Objectif du projet}
	 Ce projet a pour objectif de réaliser un distributeur automatique de nourriture pour chat contrôlé à distance via une application mobile (Android) développée en Kotlin, utilisant un microcontrôleur (ESP32), plusieurs capteurs, un servo-moteur pour la partie matérielle et Firebase comme base de données en temps réel pour la synchronisation des données.

\section{Besoins des utilisateurs}

Voici la liste des besoins des utilisateurs :

\begin{itemize}
\item     Programmation des repas : Définir des horaires et des quantités de nourriture.

\item     Contrôle manuel : Distribuer de la nourriture à la demande via l'application.

\item     Notifications : Alertes lorsque la nourriture est distribuée ou lorsque le niveau est faible.

\item     Historique : Suivi des distributions passées.

\item     Sécurité : Authentification des utilisateurs pour éviter les accès non autorisés.

\item  Réglage des quantités de quantité de nourriture pour chaque distribution

\item 
\end{itemize}
    
\section{Ressource nécessaire}

\pagenumbering{arabic}
\setcounter{page}{1}       






\chapter{Conception logicielle}
\addcontentsline{toc}{chapter}{Conception logicielle}

\section{Architecture logicielle}
L’application suit une architecture modulaire basée sur le MVVM(modèle-vue-vue modèle), divisée en trois couches principales :

\begin{itemize}
\item    Présentation (UI) : Gérée par des Composants Jetpack Compose et des ViewModels.
\item    Domain (Métier) : Contient les Use Cases et les interfaces de Repository.
\item    Data (Accès aux données) : Implémente les sources de données (Firebase, ESP32, Capteurs).
\end{itemize}
Pour des soucis de performance, les données envoyés aux microcontrôleurs sont des données de types booléennes, pour que l'éxecution se fait approximativement en temps réel. 


\section{Structure de la base de données dans Realtime Database}

L'application utilise une base de données de type \textit{clé-valeur} appelée \textbf{Realtime Database}, hébergée et gérée par Firebase. Cette base de données offre plusieurs avantages, tels qu'une infrastructure sans serveur, une haute disponibilité et une sécurité assurée par Google. La structure des données utilisée est illustrée à la figure~\ref{fig:structure_de_la_base_de_donnees}.

\begin{figure}[H]
   \centering
   \includegraphics[scale=0.5]{dd3.jpeg}
   \caption{Structure de la base de données}
   \label{fig:structure_de_la_base_de_donnees}
\end{figure}

La base de données contient deux clés principales :

\begin{itemize}
\item \textbf{users} : contient les informations de tous les utilisateurs de l'application mobile.
\item \textbf{distributors} : contient les paramètres des distributeurs automatiques de nourriture pour chats.
\end{itemize}

\begin{enumerate}[label=\alph*)]
\item \textbf{Données des utilisateurs} 

\textbf{Firebase Authentication} est utilisé pour gérer l'authentification des utilisateurs. Les informations d'authentification sont stockées et gérées par ce service. Les données spécifiques à chaque utilisateur, stockées dans Realtime Database, sont :

\begin{itemize}
\item \textbf{\{userId\}} : identifiant unique généré par Firebase Authentication pour chaque utilisateur.
\item \textbf{email} : adresse e-mail utilisée par l'utilisateur pour se connecter à l'application.
\end{itemize}

\item{Données des distributeurs} 

Les données des distributeurs servent principalement à configurer leurs actions. Ces données incluent :

\begin{itemize}
\item \textbf{\{distributorId\}} : identifiant unique de chaque distributeur, généré lors de l'assemblage du matériel.
\item \textbf{assignedTo} : identifiant de l'utilisateur associé au distributeur.
\item \textbf{status} : état de la connexion Internet de l'ESP32.
\item \textbf{lastUpdate} : date de la dernière activité du distributeur.
\item \textbf{capacity} : capacité totale du réservoir en grammes.
\item \textbf{settings} : paramètres de distribution des croquettes, comprenant :
  \begin{itemize}
    \item \textit{quantity} : quantité à distribuer en grammes.
    \item \textit{criticalThreshold} : seuil critique du réservoir en grammes.
  \end{itemize}
\item \textbf{currentWeight} : poids actuel mesuré du réservoir en grammes.
\item \textbf{planning} : liste des distributions programmées par l'utilisateur. Chaque entrée contient :
  \begin{itemize}
    \item \textit{time} : heure de distribution au format \textit{HH:mm}.
    \item \textit{enabled} : indique si la distribution est active.
    \item \textit{break} : indique une pause dans la distribution.
  \end{itemize}
\item \textbf{triggerNow} : booléen indiquant un déclenchement manuel du distributeur.
\item \textbf{history} : historique des distributions effectuées.
\end{itemize}



\end{enumerate}

\section{Fonctionalité clés de l’application} 

\subsection{Gestion des utilisateurs (Firebase Auth)}
L'inscription et la connexion se fait via email et mot de passe. Seul l’utilisateur authentifié peut contrôler son distributeur.

\subsection{Programmation des repas} 
Les horaires de distributions et la quantité sont stockés dans Firebase avec \textbf{Real Time Database}.

\subsection{Contrôle manuel}
On envoi une commande depuis Firebase qui provoque un déclenchement immédiat sur l’ESP32. Un "trigger" de type booléen qui permet au microcontrôleur d'effectuer  ou non une action sur l'ensemble du système.

\subsection{Notifications}
Des alertes sont envoyés, comme la confirmation de la nourriture distribuée et le niveau de nourriture faible. Ces notifications viennent de l'application mobile, qui envoi un "trigger"



\chapter{Conception électronique}
%\addcontentsline{toc}{chapter}{Conception électronique}




\pagestyle{fancy}
\fancyhead{} % clear all header fields
\fancyhead[LO,CE]{Réalisation}
\fancyhead[RO,LE]{2024-2025}

\chapter{Réalisation du projet}
\section{Mise en place de l'environnement de développement}
\subsection{Configuration de Firebase Realtime Database}
Dans Firebase, tout est organisé en projets afin de pouvoir regrouper les ressources et de faciliter la gestion des règles et des politiques de sécurité. La première étape consiste donc à créer un projet dans Firebase, comme illustré sur la figure~\ref{fig:creation_projet_dans_firebase}.

\begin{figure}[H]
   \centering
   \includegraphics[scale=0.5]{firebase-project-setup.png}
   \caption{Création de projet dans Firebase}
   \label{fig:creation_projet_dans_firebase}
\end{figure}

Ensuite, on doit activer Firebase Authentication afin de pouvoir l'utiliser comme méthode d'authentification dans l'application mobile. Dans la console d'administration de Firebase, il faut aller dans \textbf{Créer > Authentication > Méthode de connexion > Ajouter un fournisseur > Adresse e-mail/Mot de passe}, puis activer la fonctionnalité comme illustré dans la figure~\ref{fig:activation_de_Firebase_Auth}.

\begin{figure}[H]
   \centering
   \includegraphics[scale=0.5]{firebase_auth.png}
   \caption{Activation de Firebase Authentication}
   \label{fig:activation_de_Firebase_Auth}
\end{figure}

Pour créer l'instance de Realtime Database, il faut passer par \textbf{Créer > Realtime Database > Créer une base de données}, puis suivre les étapes de réglage des options et des règles de sécurité. Une fois activée, on obtient un lien comme illustré sur la figure~\ref{fig:db_dans_realtime_db}. 

\begin{figure}[H]
   \centering
   \includegraphics[scale=0.54]{firebase_realtime_db.png}
   \caption{La base de données dans Realtime Database}
   \label{fig:db_dans_realtime_db}
\end{figure}


\section{Développement de l'application mobile}
	\subsection{Repository}
		Les classes \verb|Repository| sont des classes d'abstraction de la communication vers la base de données. Elle sont appelé par la partie du code qui gère l'interface utilisateur pour récupérer les données. 	Dans la figure~\ref{fig:history-repository}, la classe \verb|HistoryRespository| permet de récupérer les données des historique de distributions. 
	\begin{figure}[H]
   			\centering
   			\includegraphics[scale=0.2]{history-repository.png}
   			\caption{Extrait de code d'une classe Repository}
   			\label{fig:history-repository}
	\end{figure}
	
\subsection{Interface utilisateur}
	L’interface utilisateur a été créée avec Kotlin combiné avec Jetpack Compose afin de proposer une interface conviviale, moderne et réactive.
La figure~\ref{fig:ui-code} illustre l’utilisation de Jetpack Compose pour la mise en place de l’écran d’historique. Cet extrait de code montre la structure globale de l’écran HistoryScreen, avec la gestion de l’état (par exemple, la période sélectionnée), le chargement dynamique des données depuis la base de données en arrière-plan, et l’affichage conditionnel des éléments filtrés.
On y voit également l’intégration d’un TopBar et d’un BottomNavigationBar, des filtres de période sous forme de FilterChip, ainsi que l’affichage d’une liste d’entrées via des Card, offrant une expérience utilisateur fluide et interactive.
	
	\begin{figure}[H]
   			\centering
   			\includegraphics[scale=0.22]{ui-code.png}
   			\caption{Extrait de code de l'interface utilisateur}
   			\label{fig:ui-code}
	\end{figure}
	
\section{Développement du serveur websocket}
\subsection{Besoin}
Le serveur websocket devrait être utilisé pour : 
\begin{itemize}
\item récupération des paramètres des distributeurs dans la base de données
\item détection des déclenchements manuels 
\item surveillance de l'état de connexion de l'ESP32 à internet
\item mise en place des crons pour les distributions planifiés
\item récupération des données des capteurs et les stocker dans la base de données
\end{itemize}

\subsection{Codage du serveur WebSocket}
Les étapes suivantes ont été suivies durant le développement du serveur WebSocket :
\begin{itemize}
\item Initialisation du serveur WebSocket : la bibliothèque \textbf{express-ws} a été utilisée pour l'implémentation du serveur afin d'accélérer le développement en abstrahant la gestion manuelle des messages WebSocket. La figure~\ref{fig:express-ws} illustre comment cela est mis en place.

\begin{minipage}{\linewidth}
  \centering
  \includegraphics[scale=0.25]{express-server.png}
  \captionof{figure}{Initialisation du serveur WebSocket}
  \label{fig:express-ws}
\end{minipage}
\\

\item Initialisation de la connexion vers Firebase Realtime Database : pour cela, nous créons un compte de service dans la base de données et récupérons ses identifiants de connexion sous forme de JSON, que nous chargeons dans la variable d'environnement \verb|FIREBASE_CONFIG| afin de garantir que le code ne contient pas d'informations sensibles. La figure~\ref{fig:firebase-initialization} montre comment nous utilisons cette variable d'environnement pour initialiser la connexion vers Firebase Realtime Database.

\begin{minipage}{\linewidth}
  \centering
  \includegraphics[scale=0.3]{firebase-initialization.png}
  \captionof{figure}{Initialisation de la connexion vers Firebase Realtime Database}
  \label{fig:firebase-initialization}
\end{minipage}
\\

\item Récupération des données : la récupération asynchrone des données se fait en ciblant la hiérarchie du JSON dans la fonction \verb|db.ref|. Ceci est illustré par l'extrait de code dans la figure~\ref{fig:recuperation-donnees}.

\begin{minipage}{\linewidth}
  \centering
  \includegraphics[scale=0.25]{recuperation-donnees.png}
  \captionof{figure}{Récupération des données}
  \label{fig:recuperation-donnees}
\end{minipage}
\\

\item Envoi des données et des commandes vers l'ESP32 : ceci se fait globalement par des messages WebSocket. On suit le format JSON pour les communications entre le serveur WebSocket et l'ESP32 :\begin{alltt}
{ "type": "<type_de_message>", "data": "<donnees_associees>" }
\end{alltt} 
Un exemple concret est montré par la figure~\ref{fig:envoi-donnees}, illustrant l'envoi des données de configuration initiale à l'ESP32.

\begin{minipage}{\linewidth}
  \centering
  \includegraphics[scale=0.2]{envoi-donnees.png}
  \captionof{figure}{Envoi des données vers l'ESP32}
  \label{fig:envoi-donnees}
\end{minipage}
\\

\item Surveillance des changements dans la base de données : pour surveiller les changements sur une clé dans la base de données, on utilise le \textbf{Firebase Data Stream} pour surveiller en temps réel les modifications de cette clé. Le fonctionnement est simple : à chaque mise à jour des données, le serveur WebSocket est notifié de ce changement et exécute le code nécessaire. La figure~\ref{fig:surveillance-donnees} en est un exemple concret.

\begin{minipage}{\linewidth}
  \centering
  \includegraphics[scale=0.3]{surveillance-changement.png}
  \captionof{figure}{Fonction pour surveiller les changements sur une clé donnée}
  \label{fig:surveillance-donnees}
\end{minipage}
\\

\item Mise en place des scripts cron : à chaque planification détectée dans la base de données, un script cron est planifié du côté NodeJS. La figure~\ref{fig:cron-task} illustre la fonction qui prend en paramètre l'heure et l'action à exécuter pour la tâche cron associée.

\begin{minipage}{\linewidth}
  \centering
  \includegraphics[scale=0.2]{cron-task.png}
  \captionof{figure}{Planification des tâches cron}
  \label{fig:cron-task}
\end{minipage}
\\
\end{itemize}


\section{Montage du système et réglages matériels}
\subsection{Capteur de pesage}
\begin{itemize}
	\item montage 
	\item callibrage
\end{itemize}


\subsection{Servo moteur}
\begin{itemize}
	\item montage
	\item configuration initialie nécessaire
\end{itemize}


\subsection{ESP32}
\begin{itemize}
	\item montage
	\item gestion des messages depuis le serveur websocket
\end{itemize}



%\chapter*{Conclusion}
\addcontentsline{toc}{chapter}{Conclusion}

Ce projet alliant Kotlin (Android), ESP32 et Firebase illustre parfaitement comment les technologies modernes peuvent simplifier et améliorer le quotidien, même pour nos animaux de compagnie. Grâce à une architecture robuste (MVVM) et une synchronisation temps réel via Firebase, le système offre un contrôle intelligent qui permet une programmation précise des repas et distribution manuelle à distance. Une réactivité : le système de notifications instantanées et l'historique des distributions. Modularité une codebase claire (couches séparées) facilitant les évolutions futures. Accessibilité  une interface utilisateur intuitive (Jetpack Compose) et sécurisée (Firebase Auth).

Perspectives d’amélioration : 
\begin{itemize}

\item  Vision par caméra (ESP32-CAM) pour surveiller l’animal.

\item    Reconnaissance faciale (IA embarquée) pour personnaliser les repas.

\item    Optimisation énergétique avec mode veille et batterie de secours.
\end{itemize} 

Ce dispositif démontre aussi comment l’IoT et le mobile peuvent s’unir pour résoudre des problèmes concrets.

%\input{résumé}
%{\let\clearpage\relax\par \input{Abstract}} %mettre 2 chap sur une même page%

\end{document}
