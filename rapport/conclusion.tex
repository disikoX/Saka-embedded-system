\chapter*{Conclusion}
\addcontentsline{toc}{chapter}{Conclusion}

Ce projet alliant Kotlin (Android), ESP32 et Firebase illustre parfaitement comment les technologies modernes peuvent simplifier et améliorer le quotidien, même pour nos animaux de compagnie. Grâce à une architecture robuste (MVVM) et une synchronisation temps réel via Firebase, le système offre un contrôle intelligent qui permet une programmation précise des repas et distribution manuelle à distance. Une réactivité : le système de notifications instantanées et l'historique des distributions. Modularité une codebase claire (couches séparées) facilitant les évolutions futures. Accessibilité  une interface utilisateur intuitive (Jetpack Compose) et sécurisée (Firebase Auth).

Perspectives d’amélioration : 
\begin{itemize}

\item  Vision par caméra (ESP32-CAM) pour surveiller l’animal.

\item    Reconnaissance faciale (IA embarquée) pour personnaliser les repas.

\item    Optimisation énergétique avec mode veille et batterie de secours.
\end{itemize} 

Ce dispositif démontre aussi comment l’IoT et le mobile peuvent s’unir pour résoudre des problèmes concrets.
